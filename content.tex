\begin{frame}
    \titlepage
\end{frame}

% ---------------------------------------------------------------------------------------------
\begin{frame}{Содержание курса}
\begin{enumerate}
    \item Введение в дисциплину. Зачем надо проводить исследования?
    \item Исследования и замеры в работающих скважинах. Построение моделей работающих скважин. Барометрия, термометрия, дебитометрия.
    \item Исследования пласта и призабойной зоны. Гидродинамические исследования. Построение простых моделей.
    \item ГДИС. Интерпретация исследований.
    \item Исследования пласта и скважин. Прокси модели пласта. Оптимизация работы скважин.
\end{enumerate}
Курс состоит из набора задач.
\end{frame}

% ---------------------------------------------------------------------------------------------
\begin{frame}{Расписание 2020 года}
Расписание отличается от прошлых лет.
\begin{itemize}
    \item 6 семинаров очно -- по 3 пары каждую неделю 
    \item Лекции онлайн (9 лекций) на \href{edu.gubkin.ru}{edu.gubkin.ru} 
\end{itemize}

Схема работы в течении семестра:
\begin{itemize}
    \item На сайте \href{edu.gubkin.ru}{edu.gubkin.ru} лекции и материалы для решения задач, а также задания
    \item Семинары -- разбор решения задач. Будет 5 блоков задач. Желательны компьютеры на семинарах. 
    \item На сайте \href{edu.gubkin.ru}{edu.gubkin.ru} оценки и отчеты о выполнении заданий
\end{itemize}

\end{frame}

% ---------------------------------------------------------------------------------------------
\begin{frame}{Курс 2020 года. Баллы}
Цель курса - научиться решать задачи связанные с исследованием скважин и пластов с использованием различных инструментов для проведения расчетов. 

Способы набрать баллы:
\begin{itemize}
    \item решение задач (от 1 до 10 баллов за задачу) -- основной способ;
    \item работа с материалами на сайте, подготовка материалов, полезные посты на форуме (от 1 до 5 баллов за активность);
    \item работа на семинарах, доклады по проведенной работе (от 1 до 3 баллов);    
\end{itemize}

\end{frame}